\documentclass[12pt, a4paper, twoside]{article}
\usepackage{pgf,tikz,pgfplots}
%\usepackage[margin=1cm]{geometry}
\pgfplotsset{compat=1.15}
\usepackage{mathrsfs}
\usetikzlibrary{arrows}
\usepackage[russian]{babel}

\usepackage{lipsum}
\pagestyle{empty}

\usepackage{listings}
\usepackage{color}

\definecolor{mygreen}{rgb}{0,0.6,0}
\definecolor{mygray}{rgb}{0.5,0.5,0.5}
\definecolor{mymauve}{rgb}{0.58,0,0.82}

\lstset{ 
			    backgroundcolor=\color{white},   % Цвет фона
			    basicstyle=\scriptsize\ttfamily, % Црифт для кода
			    breaklines=true,                 % Автоматический перенос
			    frame=single,	                 % Рамка вокруг кода
			    keywordstyle=\color{blue},       % Стил для ключевых слов
			    language=Python,                 % Язык программирования
			    morekeywords={*,...},            % Добавление кл.сл. вручную
			    numbers=left,                    % Расположение нумерации
			    numbersep=5pt,                   % Отступ до нумерации
			    numberstyle=\tiny\sffamily\color{gray}, % Стиль нумерации
			    stringstyle=\color{red},         % Стиль строковых символов
			    tabsize=4,	                     % Размер табуляции
			}


\usepackage{fancyhdr} %загрузим пакет
\pagestyle{fancy} %применим колонтитул
\fancyhf{} %очистим хидер на всякий случай
\fancyhead[LE,RO]{\thepage} %номер страницы слева сверху на четных и справа на нечетных
\fancyhead[CO]{\leftmark}
\fancyhead[LO]{\rightmark} 
\fancyhead[CE]{текст-центр-четные} 

\usepackage[]{algorithm2e}



\usepackage{multicol}
\setlength{\columnsep}{1cm} % Расстояние между колонками
			\setlength{\columnseprule}{1pt} % Толщина линии
			\def\columnseprulecolor{\color{blue}} % Цвет линии






\begin{document}
\section{Раздел}
\subsection{Подраздел}
\definecolor{wrwrwr}{rgb}{0.3803921568627451,0.3803921568627451,0.3803921568627451}
\definecolor{rvwvcq}{rgb}{0.08235294117647059,0.396078431372549,0.7529411764705882}
\begin{tikzpicture}[line cap=round,line join=round,>=triangle 45,x=1cm,y=1cm]
\clip(-17.28,-5.87) rectangle (3.92,7.83);
\draw [line width=2pt,color=wrwrwr] (-10.98,-0.11)-- (-7.08,0.97);
\draw [line width=0.4pt,color=wrwrwr] (-7.08,0.97)-- (-8.34,5.45);
\draw [line width=0.4pt,color=wrwrwr] (-8.34,5.45)-- (-10.98,-0.11);
\begin{scriptsize}
\draw [fill=rvwvcq] (-8.34,5.45) circle (1.5pt);
\draw[color=rvwvcq] (-8.18,5.8) node {$A$};
\draw [fill=rvwvcq] (-7.08,0.97) circle (2.5pt);
\draw[color=rvwvcq] (-6.92,1.4) node {$B$};
\draw [fill=rvwvcq] (-10.98,-0.11) circle (1pt);
\draw[color=rvwvcq] (-11.32,0.22) node {$C$};
\end{scriptsize}
\end{tikzpicture}

\begin{lstlisting}
#include<iostream>
			
int main() {
	std::cout << "Hello" << std::endl;
}
\end{lstlisting}
\newpage
\lstinputlisting[language=Python]{python/fourier.py}
\newpage
\begin{multicols}{3}
	\lipsum[1-3]
\end{multicols}
\newpage
\begin{algorithm}[H]
 			    \KwData{Входные данные}
 			    \KwResult{Результат}
 			    initialization\;
 			    \While{Нет результата}{
  			        БОТАТЬ!\;
  			        \eIf{Заботал}{
  			            Отдохни\;
  			        }{
  			            БОТАЙ!\;
  			        }
  			    }
			    \caption{Как жить на физтехе}
			\end{algorithm}

\end{document}